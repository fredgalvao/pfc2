%% abtex2-modelo-trabalho-academico.tex, v-1.9.2 laurocesar
%% Copyright 2012-2014 by abnTeX2 group at http://abntex2.googlecode.com/ 
%%
%% This work may be distributed and/or modified under the
%% conditions of the LaTeX Project Public License, either version 1.3
%% of this license or (at your option) any later version.
%% The latest version of this license is in
%%   http://www.latex-project.org/lppl.txt
%% and version 1.3 or later is part of all distributions of LaTeX
%% version 2005/12/01 or later.
%%
%% This work has the LPPL maintenance status `maintained'.
%% 
%% The Current Maintainer of this work is the abnTeX2 team, led
%% by Lauro César Araujo. Further information are available on 
%% http://abntex2.googlecode.com/
%%
%% This work consists of the files abntex2-modelo-trabalho-academico.tex,
%% abntex2-modelo-include-comandos and abntex2-modelo-references.bib
%%

% ------------------------------------------------------------------------
% ------------------------------------------------------------------------
% abnTeX2: Modelo de Trabalho Academico (tese de doutorado, dissertacao de
% mestrado e trabalhos monograficos em geral) em conformidade com 
% ABNT NBR 14724:2011: Informacao e documentacao - Trabalhos academicos -
% Apresentacao
% ------------------------------------------------------------------------
% ------------------------------------------------------------------------

\documentclass[
	% -- opções da classe memoir --
	12pt,				% tamanho da fonte
	openright,			% capítulos começam em pág ímpar (insere página vazia caso preciso)
	twoside,			% para impressão em verso e anverso. Oposto a oneside
	a4paper,			% tamanho do papel. 
	% -- opções da classe abntex2 --
	%chapter=TITLE,		% títulos de capítulos convertidos em letras maiúsculas
	%section=TITLE,		% títulos de seções convertidos em letras maiúsculas
	%subsection=TITLE,	% títulos de subseções convertidos em letras maiúsculas
	%subsubsection=TITLE,% títulos de subsubseções convertidos em letras maiúsculas
	% -- opções do pacote babel --
	english,			% idioma adicional para hifenização
	french,				% idioma adicional para hifenização
	spanish,			% idioma adicional para hifenização
	brazil				% o último idioma é o principal do documento
	]{abntex2}

% ---
% Pacotes básicos 
% ---
\usepackage{lmodern}			% Usa a fonte Latin Modern			
\usepackage[T1]{fontenc}		% Selecao de codigos de fonte.
\usepackage[utf8]{inputenc}		% Codificacao do documento (conversão automática dos acentos)
\usepackage{lastpage}			% Usado pela Ficha catalográfica
\usepackage{indentfirst}		% Indenta o primeiro parágrafo de cada seção.
\usepackage{color}				% Controle das cores
\usepackage{graphicx}			% Inclusão de gráficos
\usepackage{microtype} 			% para melhorias de justificação
% ---
		
% ---
% Pacotes adicionais, usados apenas no âmbito do Modelo Canônico do abnteX2
% ---
\usepackage{lipsum}				% para geração de dummy text
% ---

% ---
% Pacotes de citações
% ---
\usepackage[brazilian,hyperpageref]{backref}	 % Paginas com as citações na bibl
\usepackage[alf]{abntex2cite}	% Citações padrão ABNT

% --- 
% CONFIGURAÇÕES DE PACOTES
% --- 

% ---
% Configurações do pacote backref
% Usado sem a opção hyperpageref de backref
\renewcommand{\backrefpagesname}{Citado na(s) página(s):~}
% Texto padrão antes do número das páginas
\renewcommand{\backref}{}
% Define os textos da citação
\renewcommand*{\backrefalt}[4]{
	\ifcase #1 %
		Nenhuma citação no texto.%
	\or
		Citado na página #2.%
	\else
		Citado #1 vezes nas páginas #2.%
	\fi}%
% ---

% ---
% Informações de dados para CAPA e FOLHA DE ROSTO
% ---
\titulo{Enriquecimento de contexto com análise semântica para assistente virtual pessoal}
\autor{Frederico Costa Galvão \\ Lucas Costa Sousa Milhomem}
\local{Brasil}
\data{2018}
\orientador{Marcelo Stehling de Castro}
\coorientador{Sandrerley Ramos Pires}
\instituicao{%
  Universidade Federal de Goiás - UFG
  \par
  Escola de Engenharia Elétrica, Mecânica e de Computação
  \par
  Bacharelado em Engenharia de Computação}
\tipotrabalho{Projeto Final 2}
% O preambulo deve conter o tipo do trabalho, o objetivo, 
% o nome da instituição e a área de concentração 
\preambulo{Projeto Final 2 em defesa de graduação do curso de Engenharia de Computação pela Universidade Federal de Goiás - UFG.}
% ---


% ---
% Configurações de aparência do PDF final

% alterando o aspecto da cor azul
\definecolor{blue}{RGB}{41,5,195}

% informações do PDF
\makeatletter
\hypersetup{
     	%pagebackref=true,
		pdftitle={\@title}, 
		pdfauthor={\@author},
    	pdfsubject={\imprimirpreambulo},
	    pdfcreator={LaTeX with abnTeX2},
		pdfkeywords={abnt}{latex}{abntex}{abntex2}{trabalho acadêmico}, 
		colorlinks=true,       		% false: boxed links; true: colored links
    	linkcolor=blue,          	% color of internal links
    	citecolor=blue,        		% color of links to bibliography
    	filecolor=magenta,      		% color of file links
		urlcolor=blue,
		bookmarksdepth=4
}
\makeatother
% --- 

% --- 
% Espaçamentos entre linhas e parágrafos 
% --- 

% O tamanho do parágrafo é dado por:
\setlength{\parindent}{1.3cm}

% Controle do espaçamento entre um parágrafo e outro:
\setlength{\parskip}{0.2cm}  % tente também \onelineskip

% ---
% compila o indice
% ---
\makeindex
% ---

% ----
% Início do documento
% ----
\begin{document}

% Retira espaço extra obsoleto entre as frases.
\frenchspacing 

% ----------------------------------------------------------
% ELEMENTOS PRÉ-TEXTUAIS
% ----------------------------------------------------------
% \pretextual

% ---
% Capa
% ---
\imprimircapa
% ---

% ---
% Folha de rosto
% (o * indica que haverá a ficha bibliográfica)
% ---
\imprimirfolhaderosto*
% ---

% ---
% Inserir folha de aprovação
% ---

% Isto é um exemplo de Folha de aprovação, elemento obrigatório da NBR
% 14724/2011 (seção 4.2.1.3). Você pode utilizar este modelo até a aprovação
% do trabalho. Após isso, substitua todo o conteúdo deste arquivo por uma
% imagem da página assinada pela banca com o comando abaixo:
%
% \includepdf{folhadeaprovacao_final.pdf}
%
\begin{folhadeaprovacao}

  \begin{center}
    {\ABNTEXchapterfont\large\imprimirautor}

    \vspace*{\fill}\vspace*{\fill}
    \begin{center}
      \ABNTEXchapterfont\bfseries\Large\imprimirtitulo
    \end{center}
    \vspace*{\fill}
    
    \hspace{.45\textwidth}
    \begin{minipage}{.5\textwidth}
        \imprimirpreambulo
    \end{minipage}%
    \vspace*{\fill}
   \end{center}
        
   Trabalho aprovado. \imprimirlocal, \rule{32pt}{1pt}   de dezembro de 2018:

   \assinatura{\textbf{\imprimirorientador} \\ Orientador} 
   \assinatura{\textbf{Professor} \\ Convidado 1}
   \assinatura{\textbf{Professor} \\ Convidado 2}
   %\assinatura{\textbf{Professor} \\ Convidado 3}
   %\assinatura{\textbf{Professor} \\ Convidado 4}
      
   \begin{center}
    \vspace*{0.5cm}
    {\large\imprimirlocal}
    \par
    {\large\imprimirdata}
    \vspace*{1cm}
  \end{center}
  
\end{folhadeaprovacao}
% ---

%%%TODO revisar dedicatória com o Lucas
% ---
% Dedicatória
% ---
\begin{dedicatoria}
   \vspace*{\fill}
   \centering
   \noindent
   \textit{ Este trabalho é dedicado aos professores do curso de Engenharia de Computação que souberam manter sempre ativos em nós o interesse e a curiosidade do aprendizado constante. } \vspace*{\fill}
\end{dedicatoria}
% ---

%%%TODO redigir agradecimentos com o Lucas
% ---
% Agradecimentos
% ---
\begin{agradecimentos}
Os agradecimentos principais são direcionados ao Prof. {\imprimirorientador} e ao Prof. Sandrerley Ramos Pires.

\end{agradecimentos}
% ---

% ---
% Epígrafe
% ---
\begin{epigrafe}
    \vspace*{\fill}
	\begin{flushright}
		\textit{``A verdade raramente é pura, e nunca é simples. \\
		(Oscar Wilde - The Importance of Being Earnest)}
	\end{flushright}
\end{epigrafe}
% ---

% ---
% RESUMOS
% ---

% resumo em português
\setlength{\absparsep}{18pt} % ajusta o espaçamento dos parágrafos do resumo
\begin{resumo}
 Nos últimos anos o consumo de conteúdo virtual aumentou significativamente, entretanto este ainda não está estruturado de maneira preparada para a dimensão prevista pelos projetos de cidades inteligentes em planejamento. A reorganização da internet pelos conceitos da semântica que levará a web ao patamar de Web 3.0 (Web Semântica) está sendo colocada em prática pela Internet das Coisas e por abordagens modernas de processamento de dados, porém a tecnologia necessária para atender as demandas estipuladas ainda se encontra desafiada. Propõe-se, portanto, pesquisa e desenvolvimento de um sistema de mapeamento e gerência de informações pessoais baseado em análise semântica de conteúdo já consumido via extensões de browsers e aplicativos móveis com o objetivo de desenvolver um assistente virtual pessoal de memória e de consumo de conteúdo.

 \textbf{Palavras-chaves}: web 3. web semântica. cidades inteligentes. internet das coisas. gerência de informação pessoal. processamento de linguagens naturais.
\end{resumo}

% resumo em inglês
\begin{resumo}[Abstract]
 \begin{otherlanguage*}{english}
   In the latest years, virtual content consumption increased significantly, but that content still isn't well structured enough to handle the order of magnitude predicted by the ongoing smart city projects. The reorganization of the internet through semantic means that will take it to the level of Web 3.0 (Semantic Web) is being put into practice by the Internet of Things and by modern data processing approaches. Nevertheless, the technology needed to satisfy the demands stipulated is still being challenged. We propose, then, research and development of a system to map and manage personal information based on semantic analysis of consumed content through browser extensions and mobile applications with the intent to develop a virtual personal assistant on memory and content consumption.

   \vspace{\onelineskip}
 
   \noindent 
   \textbf{Key-words}: web 3. semantic web. smart city. internet of things. personal information management. natural language processing.
 \end{otherlanguage*}
\end{resumo}
% ---

% ---
% inserir lista de ilustrações
% ---
\pdfbookmark[0]{\listfigurename}{lof}
\listoffigures*
\cleardoublepage
% ---

% ---
% inserir lista de tabelas
% ---
\pdfbookmark[0]{\listtablename}{lot}
\listoftables*
\cleardoublepage
% ---

% ---
% inserir lista de abreviaturas e siglas
% ---
\begin{siglas}
  \item[IoT] Internet of Things
\end{siglas}
% ---

% ---
% inserir o sumario
% ---
\pdfbookmark[0]{\contentsname}{toc}
\tableofcontents*
\cleardoublepage
% ---



% ----------------------------------------------------------
% ELEMENTOS TEXTUAIS
% ----------------------------------------------------------
\textual

% ----------------------------------------------------------
% PARTE
% ----------------------------------------------------------
\part{Introdução}
% ----------------------------------------------------------

\chapter{Contexto do problema}
\section{Processamento natural de linguagens}
\section{Cidades inteligentes}
\section{Assistentes virtuais pessoais}
\section{Privacidade na Internet das Coisas}

\chapter{Introdução à solução proposta}
\section{Coleta de dados}
\section{Processamento de dados}
\section{Enriquecimento de contexto}
\section{Aplicação dos dados}


% ----------------------------------------------------------
% PARTE
% ----------------------------------------------------------
\part{Desenvolvimento}
% ----------------------------------------------------------

\chapter{Coleta de dados}
\section{Social vs Individual}
\section{Mundo digital}
\section{Mundo analógico}

\chapter{Processamento de dados}
\section{Dado útil vs dado bruto}
\section{Interpretação e extração de informação}
\subsection{Extração de metadados}
\subsection{Extração de timestamp}
\subsection{Extração de coordenadas}
\subsection{Extração de entidades}
\subsection{Extração de conhecimento}

\chapter{Enriquecimento de contexto}
\section{Relação temporal}
\section{Relação geoespacial}
\section{Relação de entidades}
\subsection{Exatidão}
\subsection{Proximidade}

\chapter{Aplicação dos dados}
\section{Recomendação de conteúdo}
\section{Assistente de memória}
\section{Assistente contextual}

% ----------------------------------------------------------
% PARTE
% ----------------------------------------------------------
\part{Resultados}
% ----------------------------------------------------------

\chapter{Enriquecimento de contexto temporal com relacionamento de entidades}

\section{Metodologia}
\section{Metodologia}

\chapter{Desafios}
\section{Privacidade}
\section{Escalabilidade}

\chapter{Possibilidades}
\section{SOLID Pods}
\section{Sidewalk Labs}

\chapter{Conclusões}

% ----------------------------------------------------------
% Finaliza a parte no bookmark do PDF
% para que se inicie o bookmark na raiz
% e adiciona espaço de parte no Sumário
% ----------------------------------------------------------
%\phantompart


% ----------------------------------------------------------
% ELEMENTOS PÓS-TEXTUAIS
% ----------------------------------------------------------
\postextual
% ----------------------------------------------------------

% ----------------------------------------------------------
% Referências bibliográficas
% ----------------------------------------------------------
\bibliography{references}

% ----------------------------------------------------------
% Glossário
% ----------------------------------------------------------
%
% Consulte o manual da classe abntex2 para orientações sobre o glossário.
%
%\glossary


%---------------------------------------------------------------------
% INDICE REMISSIVO
%---------------------------------------------------------------------
%\phantompart
\printindex
%---------------------------------------------------------------------

\end{document}
