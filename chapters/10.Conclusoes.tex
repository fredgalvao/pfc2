\chapter{Conclusões} \label{c:conclusao}

O detalhamento das etapas necessárias para a implementação de um sistema de \textbf{coleta, processamento, enriquecimento e aplicação} de dados \textbf{num cenário de cidades inteligentes}, assim como a disposição dos desafios e possíveis problemas que a área de conhecimento denominada por \textbf{análise semântica}, reforça alguns pontos importantes e premissas de uma implementação de sucesso de sistemas como este:

\begin{itemize}
    \item Multisciplinaridade: o envolvimento de diversas áreas do conhecimento, como neurociência, psicologia, engenharia social, administração, política, desenvolvimento de sistemas computacionais, automação residencial e industrial, dentre diversas outras áreas, é crucial para o bom andamento de um projeto que tem uma área de impacto tão abrangente como este.
    \item Multidimensionalidade: a análise de diversas dimensões dos dados, como as de tempo, localização e de valor semântico, é fundamental para atingir um nível de enriquecimento de contexto eficiente em descrever os diversos cenários da vida real com riqueza e precisão, e então ser capaz de inferir e fazer uso dos relacionamentos mais implícitos e detalhados possíveis.
    \item Governança algorítmica: um modelo completo e detalhado de gestão de acesso às diversas camadas de dados de uma cidade inteligente é uma premissa certa de um projeto de cidade inteligente de sucesso. Sem esse planejamento e execução cuidadoso, os problemas sociais que podem surgir possuem o poder de inviabilizar socialmente todas as etapas previstas por este trabalho.
    \item Privacidade: o aprofundamento no estudo, conscientização e regulamentação do que tange à privacidade do usuário final e cidadão de uma cidade inteligente precisa acontecer com urgência, e iniciativas como o GDPR não podem ser consideradas como burocracia desnecessária, mas sim como parte inerente ao projeto de elevação da web e das cidades a um patamar mais inteligente, e portanto, mais humano.
\end{itemize}