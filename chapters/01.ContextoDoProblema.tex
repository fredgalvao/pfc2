\chapter{Introdução ao problema} \label{c:introducao_ao_problema}

A definição de alguns conceitos é necessária para que o trabalho se situe e para que a motivação e desafios deste sejam esclarecidos.

\section{Conceitos} \label{s:conceitos}

A seguir, serão abordados alguns dos conceitos de alto nível utilizados ao longo do trabalho.

\subsection{Processamento natural de linguagens} \label{ss:processamento_natural_de_linguagens}

O Processamento de Linguagem Natural (PLN) consiste no desenvolvimento de modelos matemáticos e computacionais para processar, extrair e estruturar informações expressas em alguma língua natural. A pesquisa em PLN está voltada, essencialmente, a três aspectos da comunicação \cite{doprocessamento}:
\begin{itemize}
\item  Som: fonologia
\item  Estrutura: morfologia e sintaxe
\item  Significado: semântica e pragmática
\end{itemize}

\citeonline{doprocessamento} define as partes internas do estudo de PLN como sendo:

\begin{quote}
    \say{A \textbf{fonologia} está relacionada ao reconhecimento dos sons que compõem as palavras de uma língua. A \textbf{morfologia} reconhece as palavras em termos das unidades primitivas que a compõem. A \textbf{sintaxe} define a estrutura de uma frase, com base na forma como as palavras se relacionam nessa frase.  A \textbf{semântica} associa significado a uma estrutura sintática, em termos dos significados das palavras que a compõem. A \textbf{pragmática} verifica se o significado associado à uma estrutura sintática é realmente o significado mais apropriado no contexto considerado.}
\end{quote}

Uma enorme quantidade de páginas na web, em sua maioria de conteúdo redigido por seres humanos, faz necessário que sistemas de extração e aquisição de conhecimento possuam conhecimento de linguagens humanas \cite{inteligencia-artificial}.

\subsection{Cidades inteligentes} \label{ss:cidades_inteligentes}

O conceito de cidades inteligentes não é novo, nem exclusivo do século XX e XXI:

\begin{itemize}
    \item Na Roma Antiga, já por volta dos anos 800 a 735 BC, existiam sistemas de escoamento de esgoto e coleta sanitária complexos e extremamente eficientes, que influenciam projetos de arquitetura e infraestrutura até hoje\footnote{\url{https://en.wikipedia.org/wiki/Sanitation_in_ancient_Rome}}.
    
    \item Londres possuía, já em 1829, sistemas de transporte público que séculos depois serviriam como modelo de eficiência urbana para diversos outros centros europeus\footnote{\url{https://en.wikipedia.org/wiki/Transport_in_London}}.
    
    \item As ruínas de Uruk, na Mesopotâmia, possuíam sistemas matemáticos ricos de contabilidade e controle de equivalência de finanças que dominavam a vida urbana\footnote{\url{https://www.bbc.com/news/business-39870485}}.
\end{itemize}

A tentativa de classificar as iniciativas atuais como \textbf{cidades inteligentes} é unanimemente vaga\footnote{\url{https://www.citymetric.com/fabric/many-ways-smart-cities-are-really-very-dumb-4384}}, e é mais produtivo dizer que ao invés de produzir \textbf{cidades inteligentes} absolutas, o objetivo dos projetos modernos é de pensar e implementar atividades e fatores que possam tornar uma cidade \textbf{mais inteligente}, fazendo uso de tecnologia e de processamento de dados.

\subsection{Assistentes virtuais pessoais} \label{ss:assistentes_virtuais_pessoais}

Podemos chamar de Assistente Virtual os agentes de interface que são geralmente representados através de personagens que têm como foco melhorar a interação entre humanos e computadores, ou entre humanos e outros humanos através de uma interface assistida por máquina. Estes assistentes, na forma de personagens, são associados com frequência a processos e mecanismos de inteligência artificial movidos em grande parte pelo processamento natural de linguagens, orientados por estímulos externos a fim de responder a uma interação ou exercer uma tarefa \cite{reategui2006agentes}.

\subsection{Privacidade} \label{ss:privacidade_na_internet_das_coisas}

O portal IoTAgenda define como área de estudo a \textbf{privacidade na internet das coisas}, tal como sendo \say{as considerações especiais necessárias para proteger a informação do usuário de exposição num ambiente de IoT, no qual a quase toda entidade ou objeto físico ou lógico pode ser atribuído um identificador único e a habilidade de comunicação autônoma através da internet ou redes similares} \footnote{\url{https://internetofthingsagenda.techtarget.com/definition/Internet-of-Things-privacy-IoT-privacy}}.

\section{Objetivos} \label{s:objetivos}

Os objetivos principais deste trabalho são:

\begin{itemize}
    \item expor os conceitos relacionados à web semântica, processamento natural de linguagens e cidades inteligentes
    \item expor o estado da arte de etapas de coleta, processamento e aplicação de dados semânticos em assistentes virtuais pessoais
    \item propor estruturas de captura e processamento de dados que possibilitem o enriquecimento de contexto através de mapeamento de entidades e análise semântica multidimensional
    \item prototipar as estruturas de dados necessárias durante o enriquecimento de contexto para a implementação de um assistente virtual pessoal que seja capaz de obter resultados melhores que os sem abordagem multidimensional
\end{itemize}

\section{Estrutura do trabalho} \label{s:estrutura_do_trabalho}

Tendo como objetivo a solidificação dos conceitos e etapas envolvidos em uma cidade inteligente e no processamento de dados necessários para a implementação de um assistente virtual pessoal, este trabalho faz uma exploração superficial das etapas de um sistema completo, presente no capítulo \ref{c:introducao_a_solucao_proposta}.

Logo após, um aprofundamento em cada uma das etapas, assim como a contextualização dessas etapas nos desafios e objetivos deste trabalho, é feito nos capítulos \ref{c:coleta_de_dados}, \ref{c:processamento_de_dados}, \ref{c:enriquecimento_de_contexto} e \ref{c:aplicacao_dos_dados}.

Tendo a base conceitual e contextualização das etapas, técnicas e problemas bem definidos, uma estrutura próxima a um protótipo é exposta no capítulo \ref{c:enriquecimento_proposto}, junto com a comparação do modelo proposto com uma abordagem clássica, a fim de demonstrar os benefícios da solução aqui sugerida.

Por fim, os desafios encontrados e confirmados ao longo do projeto são dispostos no capítulo \ref{c:desafios}, as possibilidades e trabalhos futuros na área no capítulo \ref{c:possibilidades_trabalhos_futuros}, para que então o projeto seja concluído no capítulo \ref{c:conclusao}.

