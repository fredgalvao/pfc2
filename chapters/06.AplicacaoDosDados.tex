\chapter{Aplicação dos dados}

A criação de uma interface de exposição de dados explorada pelo capítulo anterior \ref{c:enriquecimento_de_contexto} servirá de combustível para inúmeras e inimagináveis aplicações no contexto final de um assistente virtual pessoal de memória e sugestão de consumo de conteúdo, e no contexto de cidades inteligentes. Neste capítulo serão exploradas as grandes áreas de aplicação desse banco de dados de conhecimento e relações, vislumbrando o que pode ser implementado e os benefícios oferecidos por estes sistemas ao usuário final ou a uma sociedade.

\section{Recomendação de conteúdo}

Talvez a primeira categoria de aplicação a ter chegado no mercado geral fazendo uso de um sistema similar ao descrito neste trabalho - de coleta, processamento, enriquecimento e aplicação de dados - foi o de recomendação de conteúdo.

A definição de \textit{recommender systems}, ou sistemas de recomendação, foi mencionada pela primeira vez em um relatório técnico em 1990, aplicado a um sistema de recomendação e clusterização de leitura de livros e documentos \cite{Karlgren931533}. Sistemas de recomendação aplicados ao varejo estão em constante evolução desde a década de 1990 \cite{twodecades:amazon}, e seu futuro depende muito da evolução da representação digital de \textit{contexto} do usuário online \cite{evolution:recommender}.

Sistemas de recomendação são utilizados em vários setores: filmes (Netflix), músicas (Spotify), notícias (Google News), livros (Amazon), artigos de pesquisa (Semantic Scholar, Research Gate), histórico de busca online (Google), tópicos de relevância social (Twitter), \textit{experts} de um assunto \cite{expertseer}, colaboradores para pesquisas \cite{collabseer}, piadas (Jester\footnote{http://eigentaste.berkeley.edu/about.html}), e tantos outros.

O modelo de enriquecimento de contexto descrito por este projeto tem o potencial de elevar ainda mais a eficiência de sistemas de recomendação de conteúdo ao atrelar as dimensões de tempo, geolocalização e metadados a eventos de consumo de conteúdo, criando uma representação digital do perfil de um usuário muito mais rica do que é possível atualmente.

\section{Assistente de memória}
\section{Assistente contextual}
\section{Cidades inteligentes}