\chapter{Introdução à solução proposta} \label{c:introducao_a_solucao_proposta}

A proposta do trabalho é explorar a implementação de um sistema de coleta, processamento, enriquecimento, e aplicação de dados em serviços de assistência virtual de memória e de consumo de conteúdo ao usuário.

As ferramentas utilizadas e descritas neste focam na plataforma Java/JVM, por questões de familiaridade, acessibilidade, eficiência e comunidade, visto que a linguagem de programação Java ainda possui imensa relevância no mundo acadêmico e no mercado.

Este capítulo introduz cada uma das etapas de maneira superficial com o objetivo de delinear o fluxo completo do sistema proposto sem se apegar aos detalhes de implementação.

A modelagem de um sistema computacional eficiente de coleta, processamento, enriquecimento e aplicação de dados pode se beneficiar de modelos já comprovados por estudos de neurociência, processos cognitivos, e psicologia experimental de processos mnemônicos.

A lei da contiguidade temporal e espacial reforça a relevância destes modelos, e a eficiência em inferência de relacionamentos entre diversos eventos atingida por eles é copiosamente comprovada pela literatura \cite{Polyn2011SemanticCA}.

\section{Coleta de dados} \label{s:coleta_de_dados}

Entende-se que a riqueza do sistema está na diversidade e quantidade de dados coletados. Considerando que os modelos de cidades inteligentes emergentes possibilitam a coleta de inúmeros fluxos de dados diferentes por dispositivos pessoais e compartilhados, fica claro que tal sistema alcançaria um nível de complexidade e alcance contextual que nenhuma outra estrutura pode oferecer. 

\section{Processamento de dados} \label{s:processamento_de_dados}

O sistema prevê que várias camadas de tratamento e processamento de dados serão necessárias para que uma cidade inteligente consiga fornecer de maneira escalável e relevante uma interface de consumo de dados às camadas posteriores do sistema.

Algoritmos de normalização e extração de dados farão com que informação irrelevante ou desnecessária seja removida ou ofuscada a fim de discernir informação de contexto.

O uso de diversas abordagens de processamento de linguagens naturais é obrigatório, visto que alguns dos canais de coleta de dados se tratam de dados textuais que quase sempre envolvem alguma entidade ou domínio que possa ser transformado em metadado temporal ou georreferenciado.

\section{Enriquecimento de contexto} \label{s:enriquecimento_de_contexto}

Apesar das outras etapas serem de extrema importância, é papel do enriquecimento de contexto agregar valor à informação e conhecimento disponibilizados à aplicação de assistente virtual além do que as ferramentas clássicas e especializadas conseguem fornecer hoje.

A identificação de relacionamentos temporais, georreferenciais, ou de entidades é o papel mais importante de toda a pilha, e um resultado satisfatório nesta etapa depende de múltiplos sistemas de bancos de dados especialistas em sintonia com a aplicação de consumo desses dados.

É importante ressaltar que a eficácia do processo de enriquecimento de contexto de múltiplas fontes de dados tem dependência relevante da qualidade do processamento e tratamento de dados feito na etapa anterior.

\section{Aplicação dos dados} \label{s:aplicacao_dos_dados}

Uma vez que uma camada de distribuição de dados e metadados, ambos bem tratados e processados, seja criada e implementada de forma escalável e bem definida sob as premissas de um modelo completo de governança algorítmica de cidades inteligentes, inúmeras aplicações podem ser implementadas com o apoio de empresas parceiras a fim de fornecer serviços ao usuário final e aumentar de forma generalizada e acessível a qualidade de vida dos habitantes da cidade em questão.

Este trabalho se limita em planejar e discorrer sobre a implementação de sistemas que não dependam de um modelo de cidade inteligente ativo, para que resultados possam ser atingidos e demonstrados independente da sua implantação.