\chapter{Desafios}



\section{Privacidade}



\section{Escalabilidade}



\section{Efeitos do mundo digital na evolução humana}

A teoria da seleção natural e processos evolutivos tem sido colocada em cheque na era digital, onde diversas habilidades do ser humano que o diferenciou competitivamente ao longo de centenas de milhares de anos estão se perdendo pela crescente dependência do homem para com as máquinas e sistemas digitais de informação. Ferramentas de assistência cognitiva, como o motor de busca Google, possuem a maior parcela nessa influência\footnote{\url{https://neurocritic.blogspot.com/2011/07/google-stroop-effect.html}} quando o problema é observado com foco nas últimas duas décadas \cite{Sparrow2011GoogleEO}. 

Vários livros foram escritos sobre a influência da tecnologia sobre os processos biológicos, psicológicos e especialmente cognitivos \cite{theshallows, theglasscage}. Não existe unanimidade: alguns autores detalham uma visão negativa da relação do homem com o processo evolutivo\footnote{\url{http://www.ucl.ac.uk/media/library/humanevolution}}; outros descrevem de maneira otimista como o ser humano tem tomado as rédias do processo evolutivo e está decidindo seu próprio destino evolutivo\footnote{\url{http://www.kurzweilai.net/evolution-and-the-internet-toward-a-networked-humanity}}\footnote{\url{https://www.nationalgeographic.com/magazine/2017/04/evolution-genetics-medicine-brain-technology-cyborg/}}